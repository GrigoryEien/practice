\documentclass[a4paper, 14pt]{extarticle}

%% Language and font encodings
\usepackage[english, russian]{babel}
\usepackage[utf8]{inputenc}

\usepackage[a4paper,top=2cm,bottom=2cm,left=2cm,right=1.5cm,margin=15mm, lmargin=30mm]{geometry}

\title{practice9}
\author{samstikhin}
\date{September 2018}

\begin{document}

\section*{Совместное распределение.}
\subsection*{Базовый}
\begin{enumerate}


\item Даны маргинальные распределения $Y = (1/2, 1/2^2, 1/2^3, 1/2^3)$, $X = (1/3,1/6,1/2)$. Найти совместное распределение вектора $(Y,X)$. $X$ и $Y$ считать независимыми.

   \item 
Двумерное распределение пары целочисленных случайных величин $\xi$ и $\eta$ задаётся с помощью таблицы

\begin{center}
\begin{tabular}{|c|c|c|c|}
\hline
 & $\xi = -1$ & $\xi = 0$ & $\xi = 1$\\
\hline
$\eta = -1$ & $1/8$ & $1/12$ & $7/24$\\ 
\hline
$\eta = 1$ & $5/24$ & $1/6$ & $1/8$\\ 
\hline
\end{tabular}

\end{center}
где в пересечении столбца $\xi = i$ и строки $\eta = j$ находится вероятность $P\lbrace{\xi = i, \eta = j\rbrace}$. Найти:

\begin{enumerate}
    \item Маргинальные распределения
    \item Мат. ожидание $E\xi$ и $E\eta$
    \item Дисперсия $D\xi$ и $D\eta$
    \item Среднеквадратичное отклонение: $\sigma(\xi)$ и $\sigma(\eta)$ 
    \item Ковариацию: $cov(\xi, \eta) $,
    \item Корреляцию: $\rho(\xi, \eta)$
    \item Информацию от выпадения $\xi=1$
    \item Энтропию $H(\xi)$
    \item Информацию от выпадения $(\xi,\eta)=(-1,-1)$
    \item Совместную энтропию $H((\eta, \xi))$
    \item Условную энтропию $H(\eta|\xi)$
\end{enumerate}
    
\item Монета выпадает орлом с вероятностью $p>0$. Пусть $\xi$ -- число подбрасываний, необходимых для достижения 1 выпадения орла. Найти производящую функцию для вероятности $\xi$, а также среднее и дисперсию $\xi$.

\end{enumerate}

\newpage
\section*{Совместное распределение.}
\subsection*{Дополнительный}

\begin{enumerate}
   \item 
Двумерное распределение пары целочисленных случайных величин $\xi$ и $\eta$ задаётся с помощью таблицы

\begin{center}
\begin{tabular}{|c|c|c|c|}
\hline
 & $\xi = 1$ & $\xi = 2$ & $\xi = 3$\\
\hline
$\eta = 1$ & $3/24$ & $2/24$ & $5/24$\\ 
\hline
$\eta =2$ & $2/24$ & $2/24$ & $3/24$\\ 
\hline
$\eta = 3$ & $3/24$ & $2/24$ & $2/24$\\ 

\hline
\end{tabular}

\end{center}
где в пересечении столбца $\xi = i$ и строки $\eta = j$ находится вероятность $P\lbrace{\xi = i, \eta = j\rbrace}$. Найти:

\begin{enumerate}
    \item (0.4)Ковариацию: $cov(\xi, \eta) $,
    \item (0.4)Корреляцию: $\rho(\xi, \eta)$
    \item (0.4)Информацию от выпадения $(\xi,\eta)=(2,3)$
    \item (0.4)Совместную энтропию $H((\eta, \xi))$
    \item (0.4)Условную энтропию $H(\eta|\xi)$
\end{enumerate}
\item Монета выпадает орлом с вероятностью $p>0$. Пусть $\xi_k$ -- число подбрасываний, необходимых для достижения $k$ выпадений орла. Найти производящую функцию для моментов $\xi_k$, а также среднее и дисперсию $\xi_k$. 
\begin{enumerate}
    \item (1б)$k=2$
    \item (2б)$k$ - произвольное положительное
\end{enumerate}

\item (1б)Рассмотрим бесконечную последовательность $a_0, a_1, ..., a_n$. Пусть случайная величина $\xi$ принимает значение $a_i$ с вероятностью $\frac{\lambda^i}{i!} \cdot e^{-\lambda}$ для некоторого заданного $\lambda$. Найдите дисперсию случайной величины $\xi^m$ для заданного $m$.

\item (0.5)В бар с целью застрелить Бессмертного Джо
за час в среднем заходит 1 снайпер и 2 ламера. Предполагая, что
каждый заходящий не уйдет, пока не пристрелит Джо, при этом ламер
попадает с вероятностью 1/4, снайпер  с вероятностью 3/4, найдите
среднее за час число выстрелов.

\item (0.5)Дана марковская цепь. 
\begin{center}
    P = $\left ( \begin{array}{cccc}
        1/2 & 1/3 & 1/6 & 0 \\
        1/4 & 1/4 & 1/4 & 1/4\\
        1/3 & 1/3 & 1/3 & 0\\
        0 & 0 & 0 & 1
    \end{array} \right )$
\end{center}
$\xi(t)$ - случайная величина, обозначающая длину пути, пройденного по марковской цепи, начиная в позиции 0. Найти $\xi(2)$


\end{enumerate}
\end{document}

