\documentclass[a4paper, 14pt]{extarticle}


\usepackage[english, russian]{babel}
\usepackage{mathtext}
\usepackage[utf8]{inputenc}
\usepackage[a4paper,top=2cm,bottom=2cm,left=2cm,right=1.5cm,margin=15mm, lmargin=30mm]{geometry}
\usepackage{amsmath}
\newcommand{\Mod}[1]{\(\mathrm{mod}\ #1)}

\title{pr2dopSol}
\author{Sam Stikhin}
\date{September 2018}

\begin{document}

\section*{Классическая вероятность. Схема Бернулли.}
\subsection*{Дополнительный}
\begin{enumerate}
\item (0.5б)Вася и Петя подбрасывают монету до тех пор пока не выпадет ``орел''. Если это происходит на нечетном шаге, то выигрывает Вася, если на четном, то Петя. Определить вероятность победы Васи.

	
\textbf{Решение:}
$p=\frac{1}{2}$ - вероятность <<орла>>, $q=\frac{1}{2}$ - вероятность <<решки>>. Эксперимент заканчивается, когда выпадает <<орел>>. Вероятность, что Вася выкинет <<орла>> на $(2k+1)$-ом шаге равна: $p q^{2k}$. Нам нужно рассмотреть все события, в которых Вася выкидывает <<орла>>. События независимы - можем их просуммировать:
$$P = \sum_{k=0}^{\infty} p q^{2k}= \frac{p}{1-q^2} = \frac{2}{3}$$



\item (0.5б)Из чисел $\{1, 2, \ldots, N\}$ случайно выбирается
	число $a$. Найти вероятность $p_N$ того, что: 
	\begin{enumerate}
	    \item число $a$ не делится ни на $a_1$ ни на $a_2$, где $a_1$ и $a_2$ -- фиксированные натуральные взаимно простые числа;
	\item число $a$
	не делится ни на какое из чисел $a_1, a_2, \ldots, a_k$, где
	числа $a_i$ -- натуральные и попарно взаимно простые.
	\end{enumerate}
	Найти $\lim_{N\rightarrow\infty} p_N$ в случаях (a) и (b).
	
\textbf{Решение:}
Обозначим начальное множество чисел как $M$. Количество чисел, делящихся на $a_i$ из $M$, равно $\Big[\frac{N}{a_i}\Big]$. Количество чисел, делящихся на все числа из множества $A = \{a_1,\ldots,a_k\}$, из $M$ равно $\Big[\frac{N}{LCM(A)}\Big]$, где $LCM(A)$ - наименьшее общее кратное числел из множества $A$. Но $LCM(A) = a_1 \cdot \ldots \cdot a_k$, так как все числа из $A$ взамно просты между собой. Итого получаем

\begin{enumerate}
    \item Чтобы найти вероятность того, что число $a$ делится на $a_1$ или $a_2$, мы должны сложить количество чисел, делящихся на $a_1$, $a_2$ и вычесть количество чисел, которые делятся на оба числа, потому что мы посчитали их дважды (успехи), а потом поделить это на общее число исходов: 
    $$p_a = \Big(\frac{\Big[\frac{N}{a_1}\Big] + \Big[\frac{N}{a_2}\Big] - \Big[\frac{N}{a_1 \cdot a_2}\Big]}{N}\Big)$$
    Очевидно, что $p_N = 1 - p_a$. Получаем:
    $$\lim_{N \to \infty} p_N = 1 - \lim_{N \to \infty}\frac{\Big[\frac{N}{a_1}\Big]}{N} - \lim_{N \to \infty}\frac{\Big[\frac{N}{a_2}\Big]}{N} + \lim_{N \to \infty}\frac{\Big[\frac{N}{a_1 \cdot a_2}\Big]}{N}$$
    Посчитаем предел:
    $$\lim_{N \to \infty}\frac{\Big[\frac{N}{c}\Big]}{N} = \lim_{N \to \infty}\frac{\frac{N}{c} + \Big\{\frac{N \% c}{c}\Big\}}{N} = 
    \lim_{N \to \infty}\Big(\frac{1}{c} + \frac{const}{N}\Big) = \frac{1}{c}$$
    Использовав выражение сверху получим:
    $$\lim_{N \to \infty} p_N = 1 - \frac{1}{a_1} - \frac{1}{a_2} + \frac{1}{a_1\cdot a_2} = \Big(1-\frac{1}{a_1}\Big)\cdot \Big(1-\frac{1}{a_2}\Big)$$
    
    \item Действуем абсолютно аналогично, только теперь, чтобы посчитать точное значение количества чисел, делящихся на $A = \{a_1, \ldots, a_k\}$ нужно применить формулу включений и исключений:
    $$p_N = 1 - \frac{\sum\limits_{A_i \in 2^A, n=|A_i|, A \neq \emptyset} (-1)^{n-1} \Big[\frac{N}{LCM(A_i)}\Big]} {N} = \sum\limits_{A_i \in 2^A, n=|A_i|} \frac{(-1)^{n}}{LCM(A_i)} = $$
    $$= \sum\limits_{A_i \in 2^A, n=|A_i|} \frac{(-1)^{n}}{\prod\limits_{a_j\in A_i}a_j} = 1 - \frac{1}{a_1} - \frac{1}{a_2} - \ldots + \frac{1}{a_1\cdot a_2} + \frac{1}{a_1\cdot a_3} \ldots = $$$$ = \prod\limits_{a_j \in A}(1-\frac{1}{a_j})$$
\end{enumerate}




\item (0.5б)Некогда  шестизначный номер
	трамвайного, троллейбусного или автобусного билета
	считался ``счастливым'', если сумма первых его трех
	цифр совпадает с суммой последних трех цифр. Найти
	вероятность получить ``счастливый'' билет.
	
	
\textbf{Решение:} 
\begin{enumerate}
    \item Количество счастливых билетов равно количеству 6 значных чисел с суммой цифр 27.\newline Построим биекцию: $abcdef \to abc(9-d)(9-e)(9-f)$. \newline$a+b+c = d+e+f \Rightarrow a+b+c+(9-d) + (9-e) + (9 - c) = 27$
    \item Количество разбить число $27$ на $6$ слагаемых равно количеству способов расставить 5 перегородок среди 27-ми единичек или числу сочетаний с повторениями $\overline C_{27}^{5} =C_{32}^{5}$.
    \item Теперь нужно убрать такие разбиения, в которых есть слагаемое больше 9, ведь в билете присутствуют только цифры. Воспользуемся формулой включений и исключений: уберем все разбиения, где есть хотя бы одно слагаемое больше 10, а потом добавим дважды убранные разбиения с двумя слагаемыми больше 10. 
    \begin{enumerate}
        \item Пусть одно слагаемое больше 10. Чтобы посчитать такие исходы разобьем 17 на 6, а потом к любому из них добавим 10. Получим: $6\cdot C_{17+5}^{5}$.
        \item Пусть 2 слагаемых больше 10, тогда разобьем число 7 на 6 слагаемых и потом увеличим любые 2 слагаемых на 10. Получим: $С_6^2 \cdot C_{7+5}^{5}$.
    \end{enumerate}
\end{enumerate}
Итого получили количество <<счастливых>> билетов:$$C_{32}^{5} - 6\cdot C_{17+5}^{5} + С_6^2 \cdot C_{7+5}^{5} = 55252$$
Чтобы получить вероятность, поделим на все возможные исходы:
$$\frac{C_{32}^{5} - 6\cdot C_{17+5}^{5} + С_6^2 \cdot C_{7+5}^{5}}{10^6}$$
	
	
\item (0.5б)Сколько чисел необходимо взять из таблицы 
	случайных чисел, чтобы с наибольшей вероятностью обеспечивалось
	появление среди них трех чисел, оканчивающихся цифрой 7?
	
\textbf{Решение:}
Так как нас интересует лишь последняя цифра, то вместо
чисел будем тянуть цифры от 0 до 9 с вероятностью $p = \frac{1}{10}$. Пусть мы взяли n цифр, тогда количество способов выбрать среди них ровно 3 семерки - это $C_n^3 p^3 q^{n-3}$ по схеме Бернулли. Действительно, мы указываем 3 позиции цифр 7, а на остальных позициях стоят любые цифры, кроме 7. Подставляя числа получаем: $$\frac{1}{10^n}C_n^3 9^{n-3}$$ Найдем максимум этой функции по целым числам
с помощью фолфрама/питона. Он достигается при $n = 29$ и $n = 30$.

\item (0.5б)Определить вероятность получения не менее 28 
	очков при трех независимых выстрелах из спортивного 
	пистолета по мишени с максимальным числом очков, равным 10,
	если вероятность получения 30 очков равна 0,008. Известно,
	что при одном выстреле вероятность получения восьми очков
	равна 0,15, а менее восьми очков -- 0,4.
	
\textbf{Решение:}
$p_{10}, p_{9}, p_{8}$ вероятность получить соответственно 10, 9, 8 очков за выстрел.  Известно, что $p_{10}^3= 0,008$. Тогда $p_{10} = 0,2$. $p_{8} = 0.15$. $p_{8>} = 0.4$. Тогда $p_{9} = 1 - 0.2 - 0.15 - 0.4 = 0.25$. Рассмотрим варианты получения 28 и более очков:
\begin{enumerate}
    \item $30 = 10+10+10$
    \item $29 = 10 + 10 + 9$
    \item $28 = 10+9+9 = 10+10+8$
\end{enumerate}
Случаи независимые, посчитаем для них схемы Бернулли и сложим:
$$p_{10}^3 + C_3^1(p_{10}^2\cdot p_{9} + p_{9}^2\cdot p_{10} + p_{10}^2\cdot p_{8}) = 0.0935$$


\item (0.5б)Проведено 20 независимых испытаний, каждое
	из которых заключается в одновременном 
	подбрасывании трех монет. Найти вероятность того, что хотя бы в одном испытании появятся три «герба».
	
\textbf{Решение:}Пусть $p=\frac{1}{2}$ - вероятность выпадения герба. Вероятность успеха в одном испытании - это $p^3$. Найдем вероятность того, что ни в одном испытании не выпало 3 герба. Это $q = (1 - p^3)^{20}$. Тогда искомая вероятность $P = 1 - q = 1 - (1 - p^3)^{20}$.

\item (0.5б)Квантовый вычислитель  выходит из строя, если перегреваются не менее пяти кубитов I типа или не менее двух кубитов II типа. Определить вероятность выхода из строя вычислителя, если известно, что перегрелось пять кубитов, причем кубиты перегреваются независимо один от другого. Каждый перегревшийся кубит с вероятностью 0,7 является кубитом первого типа и с вероятностью 0,3 -- второго типа.
	
\textbf{Решение:} Возможные варианты перегрева:
\begin{enumerate}
\item 5 кубитов 1го типа
\item 2 кубита 2го типа и 3 первого
\item 3 кубита 2го типа и 2 первого
\item 4 кубита 2го типа и 1 первого
\item 5 кубита 2го типа
\end{enumerate}
$p_1 = 0.7$, $p_2=0.3$ Распишем для всех случаев схемы Бернулли и сложим их:
$$\big(\sum_{i=0}^{5}C_5^i p_1^i p_2^{5-i} \big) - C_5^4p_1^4p_2 = 1 - 5*0.7^4*0.3$$


\end{enumerate}
\end{document}

