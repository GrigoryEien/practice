\documentclass[a4paper, 14pt]{extarticle}

%% Language and font encodings
\usepackage[english, russian]{babel}
\usepackage[utf8]{inputenc}

\usepackage[a4paper,top=2cm,bottom=2cm,left=2cm,right=1.5cm,margin=15mm, lmargin=30mm]{geometry}

\title{practice2old}
\author{samstikhin }
\date{September 2018}

\begin{document}

\section*{Классическая вероятность. Схема Бернулли.}
\subsection*{Базовый}
\begin{enumerate}
\item  Из 28 костей домино случайно выбираются две.
Найти вероятность  того, что из них можно 
составить <<цепочку>> согласно правилам игры.
\item Ребенок играет с десятью буквами разрезной азбуки: А, А, А,
Е, И, К, М, М, Т, Т. Какова вероятность того, что при случайном
расположении букв в ряд он получит слово <<МАТЕМАТИКА>>?

\item Охотник стреляет в лося с расстояния 100 м и 
	попадает в него с вероятностью 0.5. Если при первом выстреле
	попадания нет, то охотник стреляет второй раз, но с 
	расстояния 150 м. Если нет попадания и в этом случае, то охотник
	стреляет третий раз, причем в момент выстрела расстояние до
	лося равно 200 м. Считая, что вероятность попадания 
	обратно пропорциональна квадрату расстояния, определить 
	вероятность попадания в лося.

\item В урне находится $m$ шаров, из которых $m_1$ белых и $m_2$ черных
($m_1 + m_2 = m$). Производится $n$ извлечений одного шара с возвращением его (после определения его цвета) обратно в урну. Найдите
вероятность того, что ровно $r$ раз из $n$ будет извлечен белый шар.
\item Вероятность отказа каждого прибора при испытании
	равна 0,2. Сколько таких приборов нужно испытать, чтобы с
	вероятностью не менее 0,9 получить не меньше трех отказов?

\item (Задача Стефана Банаха) В двух спичечных коробках
имеется по $n$ спичек. На каждом шаге наугад выбирается коробок, и из
него удаляется (используется) одна спичка. Найдите вероятность того,
что в момент, когда один из коробков опустеет, в другом останется $k$
спичек.

\item В корзине лежит $n$ шариков. В ходе эксперимента с равными вероятностями вытаскивают шарики из корзины и кладут обратно. Эксперимент заканчивается, когда один из шариков достали $k$ раз. Определить вероятность того, что для этого
придется производить $m<2k$ вытаскиваний.

\end{enumerate}

\newpage

\section*{Классическая вероятность. Схема Бернулли.}
\subsection*{Дополнительный}
\begin{enumerate}
\item (0.5б)Вася и Петя подбрасывают монету до тех пор пока не выпадет ``орел''. Если это происходит на нечетном шаге, то выигрывает Вася, если на четном, то Петя. Определить вероятность победы Васи.

	\item (0.5б)Из чисел $\{1, 2, \ldots, N\}$ случайно выбирается
		число $a$. Найти вероятность $p_N$ того, что: 
		\begin{enumerate}
		    \item число $a$
		не делится ни на $a_1$ ни на $a_2$, где $a_1$ и 
		$a_2$ -- фиксированные натуральные взаимно простые числа;
		\item число $a$
		не делится ни на какое из чисел $a_1, a_2, \ldots, a_k$, где
		числа $a_i$ -- натуральные и попарно взаимно простые.
		\end{enumerate}
		Найти $\lim_{N\rightarrow\infty} p_N$ в случаях (a) и (b).
	\item (0.5б)Некогда  шестизначный номер
	трамвайного, троллейбусного или автобусного билета
	считался ``счастливым'', если сумма первых его трех
	цифр совпадает с суммой последних трех цифр. Найти
	вероятность получить ``счастливый'' билет.
    \item (0.5б)Сколько чисел необходимо взять из таблицы 
	случайных чисел, чтобы с наибольшей вероятностью обеспечивалось
	появление среди них трех чисел, оканчивающихся цифрой 7?
	\item (0.5б)Определить вероятность получения не менее 28 
	очков при трех независимых выстрелах из спортивного 
	пистолета по мишени с максимальным числом очков, равным 10,
	если вероятность получения 30 очков равна 0,008. Известно,
	что при одном выстреле вероятность получения восьми очков
	равна 0,15, а менее восьми очков -- 0,4.
\item (0.5б)Проведено 20 независимых испытаний, каждое
	из которых заключается в одновременном 
	подбрасывании трех монет. Найти вероятность того, что хотя
	бы в одном испытании появятся три «герба».
    \item (0.5б)Квантовый вычислитель  выходит из строя, если перегреваются не менее пяти кубитов I типа или не менее двух кубитов II типа. Определить
	вероятность выхода из строя вычислителя, если известно, что 
	перегрелось пять кубитов, причем кубиты перегреваются независимо один
	от другого. Каждый перегревшийся кубит с вероятностью 0,7
	является кубитом первого типа и с вероятностью 0,3 -- второго
	типа.
\end{enumerate}
\end{document}
