\documentclass[a4paper, 14pt]{extarticle}

%% Language and font encodings
\usepackage[english, russian]{babel}
\usepackage[utf8]{inputenc}
\usepackage[T1]{fontenc}
\usepackage{fontspec}
\usepackage{indentfirst}
\setmainfont{Times New Roman}
\usepackage{setspace}
\usepackage{float}
\usepackage[a4paper,top=2cm,bottom=2cm,left=2cm,right=1.5cm,margin=15mm, lmargin=30mm]{geometry}

\begin{document}

\section{Математическое ожидание. Дисперсия.}
\subsection{Какая-то таблица}
\begin{center}
\begin{tabular}{cccc}
Распределение & Функция вероятности & Ожидание & Дисперсия\\
Биноминальное & $C_n^kp^k(1-p)^{n-k}$ & $np$ & $np(1-p)$\\
Пуассона & $\frac{e^{-\lambda}\lambda^k}{k!}$ & $\lambda$ & $\lambda$\\
\end{tabular}
\end{center}
\subsection{Какие-то задачи}
\begin{enumerate}
\item В учебнике по терверу на $m$ задач приходится в среднем одна абсолютно нерешаемая. Чтобы принять зачет у $n$ студентов, из учебника случайным образом выбрали $n$ задач.
\begin{enumerate}
  \item Найдите вероятность того, что среди этих задач в точности $k$ абсолютно нерешаемых, а также ожидание и дисперсию количества абсолютно нерешаемых задач.
  \item Найдите все то же приближенно, пользуясь формулой Пуассона.
\end{enumerate}
Сравните ответы, получаемые в (1) и (2) при $n=3, m=2, k=1$. Что пошло не так?

\item Количество необъяснимых явлений, приходящихся на рабочий день агентов Скалли и Малдера, распределено по закону Пуассона с параметром 8. Найдите вероятность того, что завтра это число превзойдет 4.

\item Чтобы сдать экзамен по терверу, студент должен решить бесконечное число задач. Для каждой задачи вероятность того, что студент сможет решить ее, составляет $p$, причем эта величина не зависит от всех остальных задач. Как только студент окажется неспособен решить очередную задачу, он будет отчислен. Найдите ожидание и дисперсию числа успешно решенных студентом задач.

\item Случайная величина $X$ распределена по закону Пуассона с параметром $\lambda=0.8$. Необходимо:
\begin{enumerate}
\item Выписать формулу для вычисления вероятности $P(X=m)$;
\item Найти вероятность $P(1≤X<3)$;
\item Найти математическое ожидание $E(2X+5)$ и дисперсию $D(5−2X)$.
\end{enumerate}

\item Из множества 10-буквенных слов над русским алфавитом случайным образом выбирается одно $w$ (все слова равновероятны). Найдите ожидание количества подслов “ТЕРВЕР” у слова $w$. (Подсловом считается подпоследовательность, элементы которой в исходном слове идут подряд.)

\item Количество новых теорем, изучаемых студентом за день учебы, распределено по закону Пуассона с параметром $\lambda$. Найдите ожидание модуля разности между числом теорем, которые изучит студент в день текущий и день грядущий, считая эти величины независимыми.

\end{enumerate}

\subsection{Какие-то другие задачи}
\begin{enumerate}
\item (0.5)
Пусть случайные величины $\xi$ и $\eta$ независимы и распределены по закону Пуассона с параметрами $\lambda$ и $\mu$. Докажите, что величина $\xi+\eta$ тоже распределена по закону Пуассона и найдите параметр этого распределения.

\item (0.5)
При проведении опыта на распад атома атом распадается с вероятностью $p$.
Провели $n$ экспериментов. Найдите ожидание числа двух распадов подряд.

\item (0.5) Дискретная случайная величина $\xi$ принимает $k$
	положительных значений $x_1,\ldots,x_k$ с вероятностями,
	равными соответственно $p_1,\ldots,p_k$. Предполагая, что
	возможные значения записаны в возрастающем порядке,
	доказать, что
	$$\lim_{n\rightarrow\infty}\frac{E\xi^{n+1}}{E\xi^n}=x_k. $$

\item (0.5) Существует ли дискретная случайная величина с конечным вторым моментом
	и бесконечным первым моментом?

\item (0.5) Согласно законам о трудоустройстве в городе М, наниматели обязаны предоставить всем рабочим выходной, если хотя бы у одного из них
	день рождения, и принимать на службу рабочих независимо от их дня
	рождения. За исключением этих выходных, рабочие трудятся весь год
	из 365 дней. Условимся что день рождения рабочего выбирается рав-
	новероятно из 365 дней. Работодатель максимизировал среднее число
	трудовых человеко-дней в году (то есть произведение числа рабочих
	на число трудовых дней). Сколько рабочих трудятся на фабрике в
	городе М?

\end{enumerate}

\end{document}


