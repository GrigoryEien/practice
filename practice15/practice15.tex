\documentclass[a4paper, 14pt]{extarticle}
%% Language and font encodings
\usepackage[english, russian]{babel}
\usepackage[utf8]{inputenc}

\usepackage[a4paper,top=2cm,bottom=2cm,left=1cm,right=1cm,margin=5mm, lmargin=10mm]{geometry}

\title{practice15}
\author{samstikhin}
\date{September 2018}

\begin{document}

\section*{Совместное распределение.Условное матожидание}
\subsection*{Классная работа}
Важные формулы:
$$F_{\xi,\eta}(x,y) = P(\xi \leq x, \eta \leq y) = \int_{-\infty}^{x}\int_{-\infty}^{y}\rho_{\xi,\eta}(x,y)dxdy$$
$$\rho_{\eta}(x) = \frac{\rho_{\xi}(g^{-1}(x)}{|g'(g^{-1}(x))|}~~~~~\eta = g(\xi)~~~~~g - monotonic$$
$$\rho_{\xi|\eta}(x|y_0) = \frac{\rho_{\xi,\eta}(x, y_0)}{\rho_{\eta}(y_0)}$$
$$F_{\xi|\eta}(x|y) = \int_{-\infty}^{x} (\frac{\int_{-\infty}^{y}\rho_{\xi,\eta}(u,v)dv}{\int_{-\infty}^{y}\rho_{\eta}(w)dw})du = 
\int_{-\infty}^{x}\rho_{\xi|\eta}(u,y)du$$
$$\rho_{\xi}(x) = \int_{-\infty}^{\infty} \rho_{\xi|\eta}(x|y)\rho_{\eta}(y)dy$$
$$E(X|Y)(y) = \int_{-\infty}^{\infty}x\rho_{X|Y}(x|y)dx$$
\begin{enumerate}
\item Пусть совместная плотность случайного вектора $(\xi, \eta)$ равна:
$$\rho_{\xi,\eta}(x,y) =\left\{
	\begin{array}{cc}
	xe^{-x(y+1)}, & 0\leq x,y\\
	0, & otherwise
	\end{array}\right.$$
Найти:
\begin{enumerate}
\item одномерные(маргинальные распределения) $\xi$ и $\eta$
\item условные плотности $\xi$ по $\eta$ и $\eta$ по $\xi$
\item $E(\xi|\eta)$, $E(\eta|\xi)$
\end{enumerate}

\item Пусть $\xi$ и $\eta$ — независимые случайные величины, имеющие
равномерное распределение в отрезке [0, 1]. Найти:
\begin{enumerate}
\item $E(\xi|\xi + \eta)$
\item $E(\xi^2 - \eta^2 |\xi + \eta)$
\end{enumerate}

\item Пусть случайная величина $\xi$ имеет стандартное нормальное
распределение. Найти $E(\xi|\xi^2)$
\end{enumerate}
\newpage

\section*{Совместное распределение.Условное матожидание}
\subsection*{Домашняя работа}
\begin{enumerate}
\item Пусть $\xi$ и $\eta$ — независимые случайные величины, имеющие
равномерное распределение в отрезке [0, 1]. Найти:
\begin{enumerate}
\item (0.5)$E(\xi - \eta|\xi + \eta)$
\item (0.5)$E(\xi|\xi + 2\eta)$
\end{enumerate}
\item (0.5)Найти $E(\xi|\eta)$, если совместная плотность случайного вектора
$(\xi, \eta)$ равна:
$$\rho_{\xi,\eta}(x,y) =\left\{
	\begin{array}{cc}
	4xy, & 0\leq x\leq 1, 0\leq y\leq 1 \\
	0, & otherwise
	\end{array}\right.$$

\item Пусть случайная величина $\xi$ имеет показательное распределение 
с параметром 1, а t > 0. Найти:
\begin{enumerate}
\item (0.5)$E(\xi| min(\xi, t))$
\item (0.5)$E(\xi| max(\xi, t))$
\end{enumerate}
\item (2)Пусть независимые случайные величины $\xi$ и $\eta$ имеют стандартное
 нормальное распределение. Найти $E(\xi^2 + \eta^2 |\xi + \eta)$.
\item (1)Найти $E(\xi|\eta)$, если совместная плотность случайного вектора
$(\xi, \eta)$ равна:
	$$\rho_{\xi,\eta}(x,y) =\left\{
	\begin{array}{cc}
	\frac{1+9x^2y^2}{8}, & -1\leq x,y\leq 1\\
	0, & otherwise
	\end{array}\right.$$

\item Пусть $\xi_1, \ldots \xi_n$ — независимые случайные величины, имеющие
 равномерное распределение в отрезке $[0, 1]$. Найти:
\begin{enumerate}
\item (0.5)$E(\xi_1 | max(\xi_1, \ldots \xi_n))$
\item (0.5)$E(\xi_1 | min(\xi_1, \ldots \xi_n))$
\end{enumerate}

\end{enumerate}	

\end{document}


