\documentclass[a4paper, 14pt]{extarticle}
%% Language and font encodings
\usepackage[english, russian]{babel}
\usepackage[utf8]{inputenc}

\usepackage[a4paper,top=2cm,bottom=2cm,left=2cm,right=1.5cm,margin=15mm, lmargin=30mm]{geometry}

\title{practice13}
\author{samstikhin}
\date{September 2018}

\begin{document}

\section*{Матожидание. Дисперсия.}
\subsection*{Классная работа}
\begin{enumerate}
\item Случайная величина $\xi$ имеет экспоненциальное
	распределение с параметром $\alpha$: $Р\{\xi\leq x\} =1- \exp(-\alpha x)$.
	Найти плотности распределения случайных величин:
	\begin{enumerate}
	\item 
	$ \eta_1=\sqrt{\xi}$;
	\item $\eta_2=\xi^2$; 
	\item 	$\eta_3=\frac{1}{\alpha}\ln \xi$.
	\end{enumerate}
\item Найдите медиану, математическое ожидание и дисперсию случайной величины $\xi$, имеющей экспоненциальное распределение с показателем a>0. 
\item На смежные стороны единичного квадрата равновероятно ставят по одной точке. Найти матожидание и дисперсию расстояния между ними.  
\item Вероятность сервера выйти из строя за время $\Delta t$ равна $\lambda\Delta t+o(\Delta t)$. Через $N$ лет сервер заменяют независимо от того вышел он из строя или нет. Найти среднее время работы сервера.   
\end{enumerate}

\newpage

\section*{Матожидание. Дисперсия.}
\subsection*{Домашняя работа}
\begin{enumerate}
\item Для случайной величины из классного задания 1 найти плотности распределения случайных величин:
	\begin{enumerate}
		\item (0.5б)$\eta_4=\{\xi\}$, где $\{z\}$ -- дробная часть числа $z$; 
		\item (0.5б)$\eta_5=1-e^{\alpha\xi}$.
	\end{enumerate}
\item (1б)Случайная величина $\xi$ имеет распределение Парето с показателем a>0, если плотность ее распределения задается формулой
$$\rho_\xi(x) =\left\{
	\begin{array}{cc}
	ax^{-a-1}, & x\geq 1 \\
	0, & x<1
	\end{array}\right.$$
Пусть a>2. Найдите математическое ожидание и дисперсию $\xi$.
\item (1б)Пусть $\xi$ имеет нормальное распределение, найдите матожидание и дисперсию случайной величины $\eta = \xi^2 + \sqrt{\xi}$
\item (1б)Диаметр круга измерен приближенно. Считая, что
	его величина равномерно распределена в отрезке $[a,b]$, найти
	распределение площади круга, ее среднее значение и дисперсию.
\end{enumerate}	
\end{document}


