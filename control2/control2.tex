\documentclass[a4paper, 14pt]{extarticle}

%% Language and font encodings
\usepackage[english, russian]{babel}
\usepackage[utf8]{inputenc}
\usepackage{amsmath}

\usepackage[a4paper,top=2cm,bottom=2cm,left=2cm,right=1.5cm,margin=15mm, lmargin=30mm]{geometry}

\title{practice6}
\author{samstikhin}
\date{September 2018}

\begin{document}

\section*{Контрольная 2. ДСВ.}
\begin{enumerate}
\item (3б)Рассмотрим игру: игрок вносит в казино i монет, после каждого раунда он с вероятностью 3/5 отдает одну монету, а с вероятностью 2/5 забирает себе одну монету. Игра длится $t$ раундов. Если в какой-то момент у игрока на руках окажется $n$ монет или ни одной, то игра завершается досрочно. В конце игрок получит $2j$ монет, где $j$ - число монет, которые у него остались. Вычислите матожидание выигрыша.
\item (3б)N солдат выстроились в шеренгу. Командир смотрит на шеренгу сбоку и видит, что более высокие солдаты загораживают низких или равных, и тех становится не видно. Чему равно математическое ожидание числа солдат, которых видно сбоку?

	\item (3б)Вероятность попадания стрелка в цель 0.8. Патроны выдаются стрелку до первого промаха. Найти распределение случайной величины $\xi$ равной числу выданных патронов. Определить наивероятнейшее и среднее число выданных патронов.
	\item (3б)Пусть случайные величины $\xi$ и $\eta$ независимы и одинаково распределены, причем $P(\xi = 1) = p$, $P(\xi = 0) = 1-p$. Введем новую случайную величину $$\zeta = \begin{cases} 0& \text{: $(\xi + \eta) ~mod ~2 = 0$} \\ 1& \text{: $(\xi+\eta) ~mod ~2 = 1$} \end{cases}$$
	При каких $p$ $\xi$ и $\zeta$ будут независимыми?
	\item Пусть независимые случайные величины $\xi$, $\eta$, $\zeta$ имеют одинаковое геометрическое распределение с параметром $p$. Найти:
	\begin{enumerate}
	    \item (1б) $P(\xi=\eta)$
	    \item (1б) $P(\xi\geq\eta)$
	    \item (1б) $P(\xi+\eta\leq\zeta)$
	\end{enumerate}
	\item (3б)Пусть $p_n$ - вероятность того, что число успехов в n испытаниях бернулли делится на 3. Найти реккурентное соотношение для $p_n$, а из него - производящую функцию.



\newpage
 \item 
(12б)Двумерное распределение пары целочисленных случайных величин $\xi$ и $\eta$ задаётся с помощью таблицы

\begin{center}
\begin{tabular}{|c|c|c|c|}
\hline
 & $\xi = -1$ & $\xi = -2$ & $\xi = -3$\\
\hline
$\eta = 1$ & $ 0 $ & $3/25$ & $5/25$\\ 
\hline
$\eta = 2$ & $4/25$ & $3/25$ & $4/25$\\ 
\hline
$\eta = 3$ & $3/25$ & $1/25$ & $2/25$\\ 

\hline
\end{tabular}

\end{center}
где в пересечении столбца $\xi = i$ и строки $\eta = j$ находится вероятность $P\lbrace{\xi = i, \eta = j\rbrace}$. Найти:

\begin{enumerate}
    \item (1)Маргинальные распределения $\xi$, $\eta$
    \item (1)Мат. ожидание $E\xi$ и $E\eta$
    \item (1)Дисперсия $D\xi$ и $D\eta$
    \item (1)Среднеквадратичное отклонение: $\sigma(\xi)$ и $\sigma(\eta)$ 
    \item (1)Ковариацию: $cov(\xi, \eta) $,
    \item (1)Корреляцию: $\rho(\xi, \eta)$
    \item (1)Информацию от выпадения $(\xi,\eta)=(-1,3)$
    \item (1)Энтропию $H(\xi)$
    \item (1)Совместную энтропию $H((\xi, \eta))$
    \item (1)Условную энтропию $H(\xi|\eta)$
    \item (1)Условное матожидание $Е(\xi|\eta)$
    \item (1)Условную дисперсию $D(\xi|\eta)$
\end{enumerate}

\item(3б)(Вместо 6, которая оказалось слишкои сложной). 
Пусть k шариков случайно(равновероятно) размещаются в n
ящиках. Найти среднее значение числа пустых ящиков.    


\end{enumerate}
\end{document}

