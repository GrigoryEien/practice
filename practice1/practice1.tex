\documentclass[a4paper, 14pt]{extarticle}

\usepackage[english, russian]{babel}
\usepackage[utf8]{inputenc}
\usepackage[a4paper,top=2cm,bottom=2cm,left=2cm,right=1.5cm,margin=15mm, lmargin=30mm]{geometry}

\title{practice1old}
\author{Sam Stikhin}
\date{September 2018}

\begin{document}

\section{Ликбез по ДМ. Классическая вероятность.}
\subsection{Базовый}
\begin{enumerate}
\item В ящике лежит 100 флажков: красных, зелёных,
жёлтых, синих. Какое наименьшее
число флажков надо взять не глядя, чтобы
среди них нашлось не менее 10 одноцветных?
\item Найти коэффициент при $x^{16}$ для $(1 + x^2)^{10}$.
\item Доказать формулу: $C_{n}^k = C_{n-1}^k + C_{n-1}^{k-1}$
\item Доказать формулу сочетаний с повторениями: 
$$\overline{C_n^k} = C_{n+k-1}^k$$
\item Сколькими способами можно составить хоровод из n девушек?
\item Монета брошена 2 раза (3 раза, n раз). Найти вероятность того,
	что хотя бы один раз появится орел.
\item Брошены две игральные кости. Найти вероятность, 
что сумма выпавших очков равна 7 (9 очков, 12 очков).


\item  На полке в случайном порядке расставлено
40 книг, среди которых находится трехтомник А. С. 
Пушкина. Найти вероятность того, что эти тома стоят в 
порядке возрастания слева направо (но не обязательно рядом).

\item Партия продукции состоит из десяти изделий, среди которых два
изделия дефектные. Какова вероятность того, что из пяти отобранных
наугад и проверенных изделий:\begin{enumerate}
	\item ровно одно изделие дефектное;
	\item ровно два изделия дефектные;
	\item хотя бы одно изделие дефектное?
\end{enumerate}

\item (Парадокс дней рождений) Найти вероятность того, что в классе из 23 человек, не менее двух учеников родились в один день.
\end{enumerate}
\newpage

\section{Ликбез по ДМ. Классическая вероятность.}
\subsection{Дополнительный}
\begin{enumerate}
\item (0.5б) Докажите, что среди чисел, записываемых только единицами, есть число, которое делится на 2017.
\item (0.5б) Найти коэффициент при $x^{29}$ для $(1 + x^5 + x^7 + x^9)^{100}$.
\item (0.5б) Доказать что:
$\sum_{k=0}^{n} \big(C_n^k\big)^2 = C_{2n}^n$
\item (1б) (Задача о супружеских парах) Сколькими способами $n$ супружеских пар $(n \geq 3)$можно разместить за круглым столом так, чтобы мужчины и женщины чередовались, но супруги не сидели рядом.
\item (1б) Найти число циклических последовательностей длины 7 из 4 элементов.
\item (0.5б) Дать формулы вероятности $P_n$ того, что среди тринадцати карт, извлеченных из 52 карт,
$n$ карт окажутся пиковой мастью.
\item (0.5б) В программе к экзамену по теории вероятностей 75 вопросов. 
Студент знает 50 из них. В билете 3 вопроса. Найдите вероятность того, 
что студент знает хотя бы два вопроса из вытянутого им билета. 
\item (0.5б) У театральной кассы стоят в очереди $2n$ человек. Среди
	них $n$ человек имеют лишь банкноты по 1000 рублей, а остальные —
	только банкноты по 500 рублей. Билет стоит 500 рублей. Каждый покупатель
	приобретает по одному билету. В начальный момент в кассе нет денег.
	Чему равна вероятность того, что никто не будет ждать сдачу?

\item (1б) Как-то раз 3 ковбоя Хороший, Плохой и Злой не поделили девушку низкой социальной ответственности в одном кабаре и решили устроить дуэль. Хороший попадает в цель с вероятностью $p_1$, Плохой с вероятностью $p_2$, а Злой с вероятностью $p_3$. Каждый выбирает цель с наибольшей меткостью из оставшихся (суицид никто совершать не будет). Сначала стреляет Хороший, потом Плохой, потом Злой и потом опять Хороший и.т.д. Какова вероятность Хорошему парню выйграть дуэль и заполучить девушку.
\item (0.5б)В ящике находится $m$ белых и $n$ черных шаров. Если достают черный шар, то его возвращают и кладут еще один черный шар. Если вытащен белый шар, то процесс прекращается. Определить вероятности того, что белый шар вытащили на четном и нечетном шаге.

\end{enumerate}



\end{document}


