\documentclass[a4paper, 14pt]{extarticle}

\usepackage[english, russian]{babel}
\usepackage[utf8]{inputenc}

\usepackage[a4paper,top=2cm,bottom=2cm,left=2cm,right=1.5cm,margin=15mm, lmargin=30mm]{geometry}

\title{practice3}
\author{Anton Lipin}
\date{September 2018}


\begin{document}

\section*{Независимые события. Условная вероятность.}
\subsection*{Базовый}
\begin{enumerate}
\item В подвале были заперты один белый котенок, два черных котенка, три белых щенка и четыре черных щенка. Когда открыли дверь, одно из животных сбежало. Найдите вероятность того, что сбежавшее животное:
\begin{enumerate}
  \item котенок, если оно белое.
  \item черное, если это щенок.
  \item белое, если $2\times 2=4$.
  \item щенок, если это белый щенок.
\end{enumerate}

\item Бросили две шестигранные игральные кости. Какова вероятность, что на первой выпадет шестерка, если:
\begin{enumerate}
  \item на второй выпала единица.
  \item сумма выпавших чисел семь.
  \item сумма выпавших чисел девять.
  \item разность выпавших чисел четыре.
\end{enumerate}

\item Какова вероятность того, что 2 карты, вынутые из колоды в 36 карт, окажутся одной масти?

\item В урну, содержащую $n$ шаров, опущен белый шар, после чего наудачу извлечен один шар. Найти вероятность того, что извлеченный шар окажется белым, если равновозможны все возможные предположения о первоначальном составе шаров (по цвету).

\item Из урны, содержащей a белых и b черных шаров, извлекается наугад один
шар и откладывается в сторону. Какова вероятность того, что извлеченный
наугад второй шар окажется белым, если:
\begin{enumerate}
  \item первый извлеченный шар белый;
  \item цвет первого извлеченного шара остается неизвестным?
\end{enumerate}

\item Когда Архип врет, в Англии идет дождь. Иначе дождь в Англии идет лишь с вероятностью $\frac{1}{2}$. Сэр Джон, не знакомый с Архипом, знает, что дождь в Англии идет с вероятностью $p$. С какой вероятностью врет Архип? И чему могло быть равно $p$?

\item Покажите, что если события $A$ и $B$ независимы и $A\subseteq B$, то или $P(A)=0$, или $P(B)=1$.

\item Из урны, в которой находится 6 белых и 4 черных шара, извлекаются наудачу один за другим три шара. Найти вероятность того, что:
\begin{enumerate}
  \item все три шара будут черными;
  \item будет не меньше двух шаров черного цвета.
\end{enumerate}

\item В каждой из трех урн содержится 6 черных и 4 белых шара. Из первой
урны наудачу извлечен один шар и переложен во вторую урну, после чего из
второй урны наудачу извлечен один шар и переложен в третью урну. Найти
вероятность того, что шар, наудачу извлеченный из третьей урны, окажется
белым.

\item С утра студент Иван не мог решить, идти ли ему на пары или сладко спать до двух. Тогда он решил подбрасывать монетку, пока впервые не выпадет орел. Иван решил, что если это случится на четном броске, то он пойдет в университет, иначе продолжит спать. Известно, что у Ивана две монетки: одна правильная, другая падает орлом вверх в трех случаях из четырех. Естественно, студенту хочется воспользоваться второй (кстати, почему?), но как их различить? Иван решил подбросить обе монетки по разу и взять любую из тех, на которых выпадет наибольшее количество орлов.
С какой все-таки вероятностью Иван пойдет на пары?

\end{enumerate}
\newpage

\section*{Независимые события. Условная вероятность.}
\subsection*{Дополнительный}
\begin{enumerate}
\item
Существуют ли такие события $A$, $B$ и $C$, что:
\begin{enumerate}
\item (0б)  $A$ и $B$ независимы, $B$ и $C$ независимы, но зависимы $A$ и $C$?
\item (0.5б) $A$ и $B$ зависимы, $B$ и $C$ зависимы, но независимы $A$ и $C$?
\end{enumerate}

\item (1б)
Существуют ли такие совместно зависимые события $A_1, ..., A_n$, что любые $n-1$ из них совместно независимы?

(+0.5 балла) Может ли быть, что любые $k-1$ из $A_1, ..., A_n$ совместно независимы, но любые $k$ совместно зависимы?

\item (0.5б)
$n$ шаров, среди которых белых ровно $k$, случайным образом распределили по трем ящикам (каждый шар с равной вероятностью мог попасть в каждый ящик). Оказалось, что в первом ящике все шары белые. Какова вероятность, что во втором ящике нет ни одного белого шара?

\item (1б)
Однажды Петр нашел абсолютно симметричную монетку. Он подбросил ее один раз, после чего делал броски до тех пор, пока суммарные количества выпавших орлов и решек не сравнялись. Оказалось, что всего Петр подбросил монетку $2n$ раз. Найдите вероятность того, что впервые орел был выброшен на четвертом броске.

\item (0.5б)
Три стрелка произвели залп, причем две пули поразили мишень. Найти вероятность того, что третий стрелок поразил мишень, если вероятности попадания в мишень первым, вторым и третьим стрелками соответственно равны $0.6$, $0.5$ и $0.4$.

\item (0.5б)
Монета падает орлом с вероятностью $p$. Найдите вероятность того, что после $n$ бросков выпало четное число орлов.

\item (0.5б)
На некоторой игральной кости всякое натуральное число $n$ выпадает с вероятностью $2^{-n}$. Кость эту подбросили сначала один раз, а затем еще столько раз, сколько выпало в первый. Известно, что при этом число $k$ выпало в точности $t$ раз. Найдите вероятность, с которой изначально выпало число $m$.

\item (0.5б)
В турнире участвуют $2n$ шахматистов, из них $2m$ русских и $2k$ норвежцев ($m+k\leq n$). Для проведения первого тура шахматистов случайным образом разбили на $n$ пар. Оказалось, что хотя бы один русский шахматист будет играть с нерусским. С какой вероятностью норвежцы будут играть только с норвежцами?

\item
Последовательность событий $\{A_n\}$ такова, что при любом натуральном $n$ выполняется:
\begin{enumerate}
  \item (0.25б) $P(A_{n+1}|A_n)=\frac{1}{2}$.
  \item (0.25б) $\forall m<n: P(A_n|A_m)=\frac{1}{2}$.
\end{enumerate}
Найдите точную верхнюю грань множества вероятностей пересечений бесконечных подпоследовательностей данной последовательности.

\item (1 балл)
В автобусе $n$ мест, и все билеты проданы $n$ пассажирам. Первым в автобус заходит Рассеянный Учёный и, не посмотрев на билет, занимает первое попавшееся место. Далее пассажиры входят по одному. Если вошедший видит, что его место свободно, он занимает свое место. Если же место занято, то вошедший занимает первое попавшееся свободное место. Найдите вероятность того, что пассажир, вошедший последним, займет место согласно своему билету.

\end{enumerate}
\newpage



\end{document}


