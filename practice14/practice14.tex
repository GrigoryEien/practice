\documentclass[a4paper, 14pt]{extarticle}
%% Language and font encodings
\usepackage[english, russian]{babel}
\usepackage[utf8]{inputenc}

\usepackage[a4paper,top=2cm,bottom=2cm,left=2cm,right=1.5cm,margin=15mm, lmargin=30mm]{geometry}

\title{practice14}
\author{samstikhin}
\date{September 2018}

\begin{document}

\section*{Совместное распределение. Независимость. Условное матожидание и дисперсия.}
\subsection*{Классная работа}
\begin{enumerate}
\item Совместное распределение случайных величин $\xi$ и $\eta$ имеет плотность
$$\rho_{\xi,\eta}(x,y) =\left\{
	\begin{array}{cc}
	4(x+y), & 0\leq x\leq 1, 0\leq y\leq 1 \\
	0, & otherwise
	\end{array}\right.$$
Найдите 
\begin{enumerate}
\item $E_{\xi}$, $E_{\eta}$
\item $D_{\xi}$, $D_{\eta}$
\item $E_{\xi\eta}$
\item $cov(\xi,\eta)$.
\end{enumerate}
\item Случайные величины $\xi$ и $\eta$ независимы и нормально распределены с параметрами $\mu$ и $\sigma$ . Найти коэффициент корреляции величин
$a\xi + b\eta$ и $a\xi - b\eta$.
\item Найти коэффициент корреляции между $\xi$ и $\eta = e^{-\xi}$ , если 
$\xi$ имеет стандартное нормальное распределение $N(0,1)$.
\item Совместное распределение случайных величин $\xi$ и $\eta$ имеет плотность $\rho = 1-e^{-(x+y)}$ (x,y > 0). 
Найти:
\begin{enumerate}
	\item $E(\xi|\eta = 2)$
	\item $D(\xi|\eta = 2)$
	\item $E(\xi|\eta)$
	\item $D(\xi|\eta)$
\end{enumerate}

\end{enumerate}

\newpage

\section*{Совместное распределение. Независимость. Условное матожидание и дисперсия.}
\subsection*{Домашняя работа}
\begin{enumerate}
\item (1б) Совместное распределение случайных величин $\xi$ и $\eta$ имеет плотность
$$\rho_{\xi,\eta}(x,y) =\left\{
	\begin{array}{cc}
	\frac{3}{2}(x^2+y^2), & 0\leq x\leq 1, 0\leq y\leq 1 \\
	0, & otherwise
	\end{array}\right.$$
Найдите 
\begin{enumerate}
\item $E_{\xi}$, $E_{\eta}$
\item $D_{\xi}$, $D_{\eta}$
\item $cov(\xi,\eta)$.
\end{enumerate}
\item (1б) Пусть $X$,$Y$ - независимые случайные величины с ф.р.$FX=1-e^{-ax}$,$FY=1-e^{-by}$,$x,y,a,b>0$. Найти $E(XY)$.
\item (1б) Найти коэффициент корреляции между $\xi$ и $\eta = e^{-\xi}$ , если 
$\xi$ имеет нормальное распределение с параметрами $\mu$ и $\sigma$.
\item (1б) Доказать, что не существует трех случайных величин $\xi$, $\eta$ и
$\theta$ таких, что коэффициент корреляции любых двух из них равен -1.
\item (1б)Пусть случайные величины $\xi$ и $\eta$ имеют нулевые средние зна-
чения, единичные дисперсии
и коэффициент корреляции $c$. Показать,
что $E\max{(\xi^2,\eta^2)} \geq 1 + \sqrt{1 - c^2} $.
\end{enumerate}	

\end{document}


