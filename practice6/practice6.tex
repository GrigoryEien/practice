\documentclass[a4paper, 14pt]{extarticle}

%% Language and font encodings
\usepackage[english, russian]{babel}
\usepackage[utf8]{inputenc}

\usepackage[a4paper,top=2cm,bottom=2cm,left=2cm,right=1.5cm,margin=15mm, lmargin=30mm]{geometry}

\title{practice6}
\author{samstikhin}
\date{September 2018}

\begin{document}

\section*{6. ДСВ. Основные инструменты}
\subsection*{Базовый}
\begin{enumerate}
\item У нас есть правильный кубик и неправильная монетка, на сторонах которой написаны $0$ и $1$: $0$ выпадает с вероятностью $\frac{2}{3}$, $1$ соответственно $\frac{1}{3}$. Пусть $\xi$ - ДСВ, которая соответствует числу, выпавшему на грани первого кубика, а $\mu$ - на неправильной монетке.

Найти закон распределения, функцию распределения, матожидание и дисперсию случайной величины:
\begin{enumerate}
    \item $\theta = 2\xi +\mu +  3$
    \item $\theta = \xi\mu$
    \item $\theta = max(\xi-4, \mu)$
\end{enumerate}

\item В партии из 10 деталей имеется 8 стандартных.
	Наудачу отобраны две детали. Составить закон 
	распределения числа стандартных деталей среди отобранных.

	
\item Имеется $n$ пронумерованных писем и $n$ пронумерованных конвертов. Письма случайным образом раскладываются по конвертам (все $n!$
	способов равновероятны). Найдите математическое ожидание числа
	совпадений номеров письма и конверта (письмо лежит в конверте с
	тем же номером).
\end{enumerate}

\newpage

\section*{6. ДСВ. Основные инструменты}
\subsection*{Домашка}
\begin{enumerate}
\item У нас есть правильный кубик и неправильный кубик, в котором смещен центр тяжести и грани $3$ и $4$ выпадают с вероятностями $\frac{1}{3}$, а все остальные с вероятностями $\frac{1}{12}$. Пусть $\xi$ - ДСВ, которая соответствует числу, выпавшему на грани первого кубика, а $\mu$ - на втором кубике.

(1.5б)Найти закон распределения случайной величины:

(ФТ-301) $\theta = \xi^\mu - \mu^\xi$

(ФТ-302) $\theta = gcd(\xi^2,3 \mu)$

(КБ-301) $\theta = \min(2^\xi,\mu)$

(КН-301) $\theta = lcm(\xi+2,\mu\xi)$

(КН-302) $\theta = \max(\xi + \mu,2\mu)$

(0.5б) Нарисовать график функции распределения $\theta$

(0.5б) Найти матожидание $\theta$

(0.5б) Найти дисперсию $\theta$

Вероятности у итоговой ДСВ нужно представить в виде целочисленных дробей.

P.S. Задание, очевидно, предполагает написание программного кода. В решении нужно прислать код на любом языке программирования (только без экзотики, пожалуйста) и описание запуска, для получения ответа. Уровень красоты кода оцениваться не будет, будет оцениваться только правильность получения ответа. Предполагается, что ваш код можно легко адаптировать на решение других подобных задач (все 5 вариантов), а не на конкретном примере.

\item (1б)Из двух орудий поочередно ведется стрельба по
	цели до первого попадания одним из орудий. Вероятность
	попадания в цель первым орудием равна 0,3, вторым -- 0,7.
	Начинает стрельбу первое орудие. Составить законы 
	распределения дискретных случайных величин $\xi$ и $\eta$ -- числа
	израсходованных снарядов соответственно первым и 
	вторым орудием.

\item (1б)На первом этаже семнадцатиэтажного общежития в лифт вошли
	десять человек. Предполагая, что каждый из вошедших (независимо
	от остальных) может с равной вероятностью жить на любом из шестнадцати этажей (со 2-го по 17-й), найдите математическое ожидание
	числа остановок лифта.
\end{enumerate}
\end{document}

