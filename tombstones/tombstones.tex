\documentclass[a4paper, 14pt]{extarticle}

%% Language and font encodings
\usepackage[english, russian]{babel}
\usepackage[utf8]{inputenc}
\usepackage{amsmath}
\usepackage{amsfonts}
\usepackage{amssymb}
\usepackage{calrsfs}
\usepackage[mathscr]{euscript}
\usepackage[a4paper,top=2cm,bottom=2cm,left=2cm,right=1.5cm,margin=15mm, lmargin=30mm]{geometry}
\title{tombstones}
\author{Sam Stikhin}
\date{September 2018}

\begin{document}
\begin{enumerate}
\section*{Гробы}
	\item  Урна содержит $N$ шаров с номерами от 1 до $N$ . Пусть $K$ --
наибольший номер, полученный при $n$ их поштучных извлечениях с
возвращением. Найдите:	\begin{enumerate}
	\item  распределение $K$;
	\item Математического ожидание $EK$ при $N\rightarrow\infty$.
\end{enumerate}
    \item Восемь мальчиков и семь девочек купили билеты в кинотеатр на
	15 подряд идущих мест. Все 15! возможных способов рассадки равно-
	вероятны. Вычислите среднее число пар рядом сидящих мальчика и
	девочки.
\item Хороший, Плохой и Злой так и не выявили победителя. Ковбои попадают с вероятностью $p_1$, $p_2$ и $p_3$ соответственно. Стреляют по очереди: сперва Хороший, затем Плохой, потом Злой, потом снова Хороший и т.д. В свою очередь каждый выбирает мишень с равной вероятностью. Найдите вероятность победы каждого из участников дуэли.
\begin{enumerate}
    \item Суицид запрещён.
    \item Суицид разрешён (вероятность выстрелить в себя равна вероятности выстрелить в одного из противников, 1/3).
\end{enumerate}
\item  Большое число $N$ людей подвергается исследованию
	крови. Это исследование может быть организовано двумя
	способами. 1. Кровь каждого человека исследуется отдельно.
	В этом случае потребуется $N$ анализов. 2. Кровь $k$ людей
	смешивается, и анализируется полученная смесь. Если  
	результат анализа отрицателен, то этого одного анализа  
	достаточно для $k$ человек. Если же он положителен, то кровь  
	каждого приходится исследовать затем отдельно, и в целом на $k$
	человек потребуется $k+1$ анализ. Предполагается, что  
	вероятность положительного результата одна и та же для всех
	людей и что результаты анализов независимы в теоретико-
	вероятностном смысле.
	\begin{enumerate}
		\item Чему равна вероятность того, что анализ смешанной
	крови k людей положителен?
	\item Чему равно математическое ожидание числа анализов, необходимых при
	втором методе исследования?
	\item При каком k достигается минимум математического ожидания числа  
	необходимых анализов?
	\end{enumerate}
 \item Дракон и Принцесса по очереди тянут мышек из мешка, в котором изначально $3$ белых и $2$ черных мышки. Выигрывает тот, кто первым достает белую мышь. После каждой вытянутой драконом мыши оставшиеся впадают в панику, и одна из них выпрыгивает из мешка сама (принцесса вытаскивает мышей из мешка аккуратно и не пугает их). Принцесса тянет первой.

Если мыши в мешке закончились, а белую так никто и не вытащил, победителем считается дракон. Мыши, которые выпрыгнули сами, не считаются вытащенными (не определяют победителя). Единожды покинув мешок, мыши в него не возвращаются. Любая мышь вытаскивается из мешка с одинаковой вероятностью, и любая мышь выпрыгивает из мешка с одинаковой вероятностью.

Изобразите марковскую цепь, описывающую данную игру. Определите вероятности победы Принцессы и Дракона.

\item Рассмотрим цепь с 3 состояниями и матрицей

\begin{center}
    P = $\left ( \begin{array}{ccc}
        1 - a & a & 0 \\
        0 & 1 - a & a \\
        a & 0 & 1 - a
    \end{array} \right ).$
\end{center}

Начальное распределение: $\left ( 1, 0, 0 \right )$. Найдите $a$ такое, что вероятность оказаться в состоянии 2 на шаге 998244353 максимальна. Если таких $a$ несколько, выберите любое.

\item Прибор Васи испускает $\alpha$, $\beta$ и $\gamma$-частицы. Прибор может работать в двух режимах, смену которых Вася не может контролировать. При работе в первом режиме прибор испускает $\alpha$, $\beta$ и $\gamma$-частицы с вероятностями $\left ( 0.41, 0.099, 0.491 \right )$ соответственно, при работе во втором: $\left ( 0.625, 0.185, 0.19 \right )$. Прибор устроен так, что после каждого пуска частицы режим меняется в соответствии с матрицей

\begin{center}
    P = $\left ( \begin{array}{cc}
        0.68 & 0.32 \\
        0.696 & 0.304
    \end{array} \right ).$
\end{center}

Вася нашёл строку $\gamma \beta \alpha \gamma \beta \gamma \alpha \alpha \alpha \gamma$ и задался вопросом: какова вероятность появления такой последовательности частиц. Помогите ему. Считайте, что начальный режим работы прибора выбран равновероятно.
	\item Пусть $N$ -- размер некоторой популяции, который требуется 
	оценить ``минимальными средствами'' без простого пересчета всех элементов
	этой совокупности. Подобного рода вопрос интересен, например, при 
	оценке числа жителей в той или иной стране, городе и т. д.
	В 1786 г. Лаплас для оценки числа $N$ жителей во Франции предложил
	следующий метод.
	Выберем некоторое число, скажем, $M$, элементов популяции и пометим
	их. Затем возвратим их в основную совокупность и предположим, что
	они ``хорошо перемешаны'' с немаркированными элементами. После этого
	возьмем из ``перемешанной'' популяции $n$ элементов. Обозначим через $X$
	число маркированных элементов (в этой выборке из $n$ элементов).
	\begin{enumerate}
		\item Показать, что вероятность $P_{N,M,n}\{X = m\}$ того, что $X=m$ задается
		(при фиксированных $N,M,n$) формулой гипергеометрического 
		распределения
		$$P_{N,M,n}\{X = m\}=\frac{C_{M}^mC_{N-M}^{n-m}}{C_N^n}. $$
		\item Считая $M$, $n$ и $m$ заданными, найти максимум $P_{N,M,n}\{X = m\}$ по $N$,
		т. е. ``наиболее правдоподобный'' объем всей популяции, приводящий к
		тому, что число маркированных элементов оказалось равным $m$.
		Показать, что так найденное наиболее правдоподобное значение $N$ 
		(называемое оценкой максимального правдоподобия) определяется 
		формулой $$\widehat{N}=\left[\frac{Mn}{m}\right], $$
		где $[\cdot]$ -- целая часть.
	\end{enumerate}
		\item Пусть $(\Omega,\mathcal{F},P)$ -- вероятностное пространство, $A_1,\ldots, A_n,\ldots\in\mathcal{F}$. Для $J\subset \{1,\ldots n\}$ положим
	$$\mathcal{P}_1(J)\triangleq \sum_{i\in J}A_i, $$
	$$\mathcal{P}_{2}(J)\triangleq \sum_{i_1,i_2\in J,i_1<i_2}P(A_{i_1}\cap A_{i_2}), $$
	$$\mathcal{P}_k(J)\triangleq \sum_{i_1,\ldots,i_k\in J,i_1<i_2<\ldots<i_k}P(A_{i_1}\cap A_{i_2} \ldots\cap A_{i_k}). $$ Доказать, что для всех нечетных $k$ справедливо неравенство
	$$P(A_{i_1}\cup\ldots A_{i_n})\leq \sum_{j=1}^k(-1)^{j+1}\mathcal{P}_j(\{i_1,\ldots,i_k\}), $$ а для всех четных справедливо неравенство
	$$P(A_{i_1}\cup\ldots A_{i_n})\geq \sum_{j=1}^k(-1)^{j+1}\mathcal{P}_j(\{i_1,\ldots,i_k\}). $$ 	\item Показать также, что алгебра, порожденная системой
	$\{A_1,\ldots, A_n\}$, где $A_i\subset\Omega$, $i = 1,\ldots, n$ состоит из $2^{2^n}$ элементов.
	\item Пусть $A_1,A_2,\ldots, A_n,\ldots$ некоторые подмножества $\Omega$, постройте минимальную $\sigma$-алгебру, включающую $A_1,A_2,\ldots, A_n,\ldots$.
	\item При проведении опыта на распад атома атом распадается с вероятностью $p$ и не распадается с вероятностью $1-p$. Найти асимптотическое значение математического ожидания и дисперсии числа появлений двух распадов подряд (мы считаем, что два распада случились подряд, если распад был на $i-1$ и $i$-м испытании).
	
\item Случайным образом выстраиваются в шеренгу $n$ человек разного
роста. Найдите вероятность того, что
\begin{enumerate}
	\item самый низкий окажется $i$-м слева;
	\item самый высокий окажется первым слева, а самый низкий -- последним слева;
	\item самый высокий и самый низкий окажутся рядом;
	\item между самым высоким и самым низким расположатся более $k$
человек.
\end{enumerate}

\item В социальной сети зарегестрировано конечное число пользователей. 
Доказать, что матожидание числа друзей у пользователя меньше или равно матожиданию матожидания числа друзей у друзей пользователя. 


\end{enumerate}

\end{document}

