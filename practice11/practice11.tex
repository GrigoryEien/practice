\documentclass[a4paper, 14pt]{extarticle}

%% Language and font encodings
\usepackage[english, russian]{babel}
\usepackage[utf8]{inputenc}

\usepackage[a4paper,top=2cm,bottom=2cm,left=2cm,right=1.5cm,margin=15mm, lmargin=30mm]{geometry}

\title{practice10}
\author{samstikhin}
\date{September 2018}

\begin{document}

\section*{Ликбез по матану. Геометрическая вероятность}
\subsection*{Базовый}
\begin{enumerate}
\item Найти
\begin{enumerate}
    \item $\int_{\frac{\pi}{3}}^{\pi} x \sin{x} dx$
    \item Площадь внутри $C = \{y = x^2, y = 2 - \sqrt{4 - x^2},x = 2, x=0\}$
    \item $\oint_C xdy + ydx$
    \item $\int_{0}^{\infty}e^{-x^2}dx$
\end{enumerate}

\item 2 лыжника условились о встрече между
	10 и 11 часами утра у подножия горы (склона), причем договорились
	ждать друг друга не более 10 минут, чтобы не замерзнуть. Считая, что
	момент прихода на встречу каждым выбирается наудачу в пределах
	указанного часа, найдите вероятность того, что встреча состоится.
	\item На плоскости, замощённой одинаковыми прямоугольниками со сторонами 10 и 20 
	(прямоугольники примыкают сторонами), рисуют случайную окружность радиуса 4. 
	Найдите вероятность того, что окружность имеет общие точки ровно с тремя прямоугольниками.


\newpage

\end{enumerate}
\section*{Геометрическая вероятность}
\subsection*{Домашка}
\begin{enumerate}
	\item (1б) Задача Бюффона. Плоскость разграфлена параллельными
	прямыми, отстоящими друг от друга на расстоянии $2a$. На плоскости
	наудачу, бросается игла длины $2l$ ($l<a$). Найти вероятность того, что
	игла пересечет какую-нибудь прямую.
    \item (1б)Найти объем параболоида $C = \{z>0,z = 1-y^2-x^2\}$через интеграл
    \item (1б)Коэффициенты $p$ и $q$ квадратного уравнения
	$$x^2 + px + q = 0$$
	выбираются наудачу в промежутке $[0,1]$. Спрашивается, чему равна 
	вероятность того, что корни будут действительными числами?
	
	\item (1б)Отрезок длины $a$ случайным образом разделен на 3 части. Найти вероятность того, 
	что длины хотя бы двоих частей не меньше, чем $a/5$.
	\item (1б)Однородный прямой круговой конус с 
	высотой $h$ и радиусом основания $r$ случайно бросается на
	горизонтальную плоскость.
	\begin{enumerate}
		\item Найти вероятность того,
	что он упадет на основание; 
		\item вычислить эту 
	вероятность при $r=h$; 
	\item при каком отношении $r/h$ эта 
	вероятность равна 1/4?
	\end{enumerate}
	\item k лыжников условились о встрече в промежуток времени $[0,T]$, причем договорились
	ждать друг друга не более $T/10$ минут. Считая, что
	момент прихода на встречу каждым выбирается наудачу в пределах
	указанного часа, найдите вероятность того, что встреча состоится.
	\begin{enumerate}
	    \item (1б) k=3
	    \item (1б) k - произвольное
	\end{enumerate}
\end{enumerate}
	

\end{document}


