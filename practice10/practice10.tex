\documentclass[a4paper, 14pt]{extarticle}

%% Language and font encodings
\usepackage[english, russian]{babel}
\usepackage[utf8]{inputenc}
\usepackage{amssymb}

\usepackage[a4paper,top=2cm,bottom=2cm,left=2cm,right=1.5cm,margin=15mm, lmargin=30mm]{geometry}

\title{practice10}
\author{samstikhin}
\date{September 2018}

\begin{document}

\section*{Условные матожидание и дисперсия.}
\subsection*{Базовый}
\begin{enumerate}

\item Доказать, что энтропия $H(f(\xi)|\xi) = 0$.

\item Пусть $\xi$ - случайная величина, отвечающая за кубик. Посчитайте $E\xi$ и $D\xi$ при условии, что у на кубике выпало или 3 или 4.

\item Пусть $\xi$ - случайная величина, отвечающая за кубик. Посчитайте функции $E(\xi|H)$ и $D(\xi|H)$, где образующее разбиение равно $H = (\{1\}, \{2,3\}, \{4,5,6\})$.

\item Посчитать индикатор $P(1_{1,2}|H)$ для выпадения 1 или 2 на кубике для разбиения $H$ из предыдущей задачи. 

\item 
Двумерное распределение пары целочисленных случайных величин $\xi$ и $\eta$ задаётся с помощью таблицы

\begin{center}
\begin{tabular}{|c|c|c|c|}
\hline
 & $\xi = -1$ & $\xi = 0$ & $\xi = 1$\\
\hline
$\eta = -1$ & $1/8$ & $1/12$ & $7/24$\\ 
\hline
$\eta = 1$ & $5/24$ & $1/6$ & $1/8$\\ 
\hline
\end{tabular}


\end{center}
где в пересечении столбца $\xi = i$ и строки $\eta = j$ находится вероятность $P\lbrace{\xi = i, \eta = j\rbrace}$. Найти:

\begin{enumerate}
    \item Таблицу условной вероятности. В каждой клетке которой находится $p_{ij} = (\xi=i|\eta=j)$.
    \item Мат. ожидание $E(\xi|\eta)$
    \item Дисперсия $D(\xi|\eta)$
\end{enumerate}

\item Доказать свойство:
    $$E(f(\eta)\xi|\eta) = f(\eta)E(\xi|\eta)$$

\end{enumerate}
\newpage
\section*{Условные матожидание и дисперсия.}
\subsection*{Базовый}
\begin{enumerate}
\item (1б)Пусть $\xi$ - случайная величина, отвечающая за неправильный кубик с вероятностями выпадения значений(по порядку) $(\frac{1}{12}, \frac{1}{12},\frac{1}{3}, \frac{1}{3},\frac{1}{12},\frac{1}{12})$. Посчитайте функции $E(\xi|H)$ и $D(\xi|H)$, где образующее разбиение равно $H = (\{1\}, \{2,3\}, \{4,5,6\})$.
\item 
Двумерное распределение пары целочисленных случайных величин $\xi$ и $\eta$ задаётся с помощью таблицы

\begin{center}
\begin{tabular}{|c|c|c|c|}
\hline
 & $\xi = -1$ & $\xi = 0$ & $\xi = 1$\\
\hline
$\eta = -1$ & $1/8$ & $1/12$ & $7/24$\\ 
\hline
$\eta = 1$ & $5/24$ & $1/6$ & $1/8$\\ 
\hline
\end{tabular}


\end{center}
где в пересечении столбца $\xi = i$ и строки $\eta = j$ находится вероятность $P\lbrace{\xi = i, \eta = j\rbrace}$. Найти:

\begin{enumerate}
    \item (0.5)Мат. ожидание $E(\eta|\xi)$
    \item (0.5)Дисперсия $D(\eta|\xi)$
\end{enumerate}
	
	\item (0.5б)Имеется $n$ пронумерованных писем и $n$ пронумерованных конвертов. Письма случайным образом раскладываются по конвертам (все $n!$
	способов равновероятны). Найдите математическое ожидание числа
	совпадений номеров письма и конверта (письмо лежит в конверте с
	тем же номером), при условии, в первые n/3 конвертов точно попали нужные письма письма.
	
	\item (0.5б)Денис играет в дартс, на котором отмечены области для всех результатов от 1 до 20. Меткость Дениса линейно возрастает с 0 для выбивания 1 и достигает максимума в 10ке, а потом линейно опускается до 0 в значении 20. Каково матожидание результата 4 бросков, если за все 4 броска Денис:
	\begin{enumerate}
	    	\item ни разу не выбил больше 8
	\item выбивал от 9 до 14
	\item только от 15 и более
	\end{enumerate}
\end{enumerate}
\end{document}


